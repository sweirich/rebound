\documentclass[sigplan]{acmart}

\usepackage{listings}
 \lstset{language=Haskell,basicstyle=\small\ttfamily}
\usepackage{url}

\settopmatter{printacmref=false}
\fancyfoot{}
\acmDOI{}
\makeatletter
\renewcommand\@formatdoi[1]{\ignorespaces}
\makeatother
\renewcommand\footnotetextcopyrightpermission[1]{}

\title{Discussion: Benchmarking Binding}
\author{Stephanie Weirich
%\orcidID{0000-0002-6756-9168}
%\affiliation{\department{Computer and Information Science}\institution{University of Pennsylvania}}
}
\email{sweirich@cis.upenn.edu}

\begin{document}
\bibliographystyle{ACM-Reference-Format}


\maketitle

\section{How should we represent binding?}

Implementations of type systems, logics and programming languages must often
answer the question of how to represent and work with binding structures in
terms. While there may not be a one-size-fits-all solution, I would like to
understand the trade-offs that various approaches to binding make in terms of
execution speed and simplicity of implementation.

Binding structures are epitomized in the lambda calculus by the abstraction
term, $\lambda x. e$, that binds the variable $x$ within the body $e$. All
occurrences of $x$ in this body are references to this bound variable. Other
forms of binders include quantifiers
(e.g. $\forall$,$\exists$,$\Pi$,$\Sigma$,$\nu$), recursive definitions,
pattern matching, etc.

To that end, I have been constructing a benchmark platform using the Haskell
programming language and have been gathering multiple implementations of
binding for comparison and
experimentation.\footnote{\url{https://github.com/sweirich/lambda-n-ways}.
This repository is based on a draft paper by Lennart Augustsson, but extends
his four original implementations of the untyped lambda calculus with many
additional variations.}
%
I propose to use this repository as a focus for a discussion about the
following questions related to aggregating and comparing implementations of
binding.

\subsection{What should a benchmark suite contain?}

We don't just representating lambda terms, we also want to work with that
representation efficiently. Therefore, the goal of a benchmark suite should be
to evaluate different implmentations of the same operations. But which ones?

The operations that I think are important include:

\begin{itemize}
 \item Conversion to / from a "raw string" representation 
%    i.e. if we may create
%    a term by parsing it from a string and we might like to display it to the
%    user using a familiar notation.
  
  \item Comparing for alpha-equivalence
%     We need an equality function that is not
%    sensitive to the names of bound variables, so that $\lambda x.x$ and
%    $\lambda y.y$.  We might also like to extend these operations to
%    well-orderings of terms or hash functions.

  \item Capture-avoiding substitution % (replacing a free variable with a term).

  \item Free variable set calculation
\end{itemize}

Designing a uniform and informative way to evaluate implementations of these
operations is challenging. One issue is that the operations have different
types with different binding representations. For example, if an
implementation of capture-avoiding substitution requires freshening, it may
require additional arguments to be sure that the new names chosen are
fresh.

But, more fundamentally, are these the right operations to begin with?  Do
implementers use capture-avoiding substitution outside of teaching students
about programming language theory? Should we add others to this list, such as
normalization, alpha-invariant ordering, or hashing? Certain representations
may also require additional functions, such as shifting (changing the scope of
a term), freshening (renaming free variables), or instantiation (replacing a
binding variable with a term) when working with terms, how do those interact
with our benchmarks?  Should we only measure larger operations, such as type
checking or source-to-source translations? How can we do that while still
comparing a large number of implementations, being sure that they all
implementing the same operation?

\subsection{What optimizations can we evaluate?}

There are ways to optimize an operation like capture-avoiding substitution
that are independent of the binding representation chosen.  For example, if we
know that a variable does not appear in a term, we can terminate the
substitution operation early. But tracking the free variables of a term comes
at a cost. Where is the trade off?

Alternatively, if we can suspend substitution operations inside terms, we can
fuse multiple traversals together, which makes a difference in, say
normalization, which requires repeated uses of substitution. Furthermore, if
substitution is interleaved with another function over lambda terms, say
$[\![a]\!]$, then instead of computing $[\![a \{b/x\} ]\!]$ it might be more
efficient to compute $[\![a ]\!]\{[\![b]\!]/x\}$ instead, when substitution
commutes with the operation.

Are there other optimizations that apply to capture-avoiding substitution? To
the other operations? How can we evaluate their effectiveness?

\subsection{What about library support?}

Several libraries exist to support Haskell programmers in implementing binding
including
\texttt{unbound-generics}\footnote{\url{https://hackage.haskell.org/package/unbound-generics}},
\texttt{bound}\footnote{\url{https://hackage.haskell.org/package/bound}}, and
others. These libraries simplify working with binding.

% For example, using \texttt{unbound-generics}, the following
% definitions automatically produce functions that implement alpha-equivalence,
% capture-avoiding substitution and open bindings with fresh variables.
% 
% \begin{lstlisting}
% data Exp = Var (Name Exp) | App Exp Exp
%   | Lam (Bind (Name Exp) Exp) 
%  deriving (Show, Generic)
%
% instance Alpha Exp where
% instance Subst Exp Exp where
%   isvar (Var x) = Just (SubstName x)
%   isvar _ = Nothing
% \end{lstlisting}

The \texttt{unbound-generics} library uses a locally nameless
representation. The bound \texttt{library} uses a variant of de Bruijn indices
based on nested datatypes. Other libraries are inspired by nominal logic. What
representations can we use for such libraries? How can we make them both easy
to use and efficient? What is the abstraction cost for introducing such a 
library? Can we design libraries that minimize this cost?

\subsection{What is used in practice?}

Above, I've provided my answers to these questions to make it clear the kinds
of discussion that I'm looking for. However, I'm keen to hear the experiences
of others, hence the discussion format.



\end{document}

Can this knowledge be wrapped up in a library to make it more accessible to
  programmers? How difficult is that library to use? At what cost to
  performance compared to hand-rolled implementation?
  
  - Minimize the amount of code that users need to write
  - Capture "scope" via monads, strong types, etc.
  - Use compile-time code generation or datatype-generic programming? (But, then what cost?)

What would a correctness proof of these implementations look like? Can we
  easily formalize them in a proof assistant as a function?1

\section{An initial proposal}

A repository to gather Haskell implementations of the untyped lambda
calculus. This repository should include documentation so that users can
browse these implementations to better understand the design space and
trade-offs. I'm proposing the untyped lambda calculus because it is small
(only three constructors, two of them related to binding) and well
studied. 

All implementations in the repository should implement the same small, core set
of operations, using a common interface. As a starting point, my existing core
set includes alpha-equivalence, conversion to/from "raw" form, and full
normalization. In the latter case, full normalization permits substitution
operations with wildly different interfaces, such as environment-based ??
However, care must be taken that all implementations provide the same
normalization function.

A suite of benchmarks that use these operations. Both in terms of functions
implemented with these operations, and


\bibliography{weirich}

\end{document}
